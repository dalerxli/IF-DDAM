\chapter{Representation of the results}\label{chap6}
\markboth{\uppercase{Representation of the results}}{\uppercase{Representation of the results}}

\minitoc

\section{Introduction}

Three windows enable us to manage and represent the requested results.
The one on the top enables us to manage the different figures; the one
at the bottom on the left present the digital values requested, and
the one at the bottom on the right is kept for the graphic
representations.

\section{Digital exits}

All the results are given in the SI system. 

\begin{itemize}

\item {\it Object subunits}: Number of elements of discretization of
  the  object under study.

\item {\it Mesh subunits} : Number of elements of discretization of
  the cuboid containing the object under study.

\item {\it Mesh size} : Size of the element of discretization.

\item $\lambda/(10|n|)$ : In order to obtain a good precision, it is
  advised to have a discretization under the value of $\lambda/10$ in
  the considered material of optical index $n$.

\item $k_0$ :Wave number.

\item {\it Irradiance}: Beam irradiance, for a Gaussian beam, it is
  estimated at the center of the waist.

\item {\it Field modulus}: Modulus of the field, for a Gaussian beam, it is 
estimated at the center of the waist.

\item {\it Tolerance obtained}: Tolerance obtained for the chosen iterative 
method. Logically under the requested value.

\item {\it Number of products Ax (iterations)}: Number of matrix
  vector products completed by the iterative method. Between brackets
  the iteration number of the iterative method.

\item {\it Absorptivity} Fraction of radiation absorbed in \%, equal
  to zero if all the permittivity are real.

\item {\it Reflectivity} Fraction of radiation reflected in \%.

  \item {\it Tansmittivity} Fraction of radiation transmitted in \%.

  
\item {\it Extinction cross section}: Value of the extinction cross
  section.

\item {\it Absorbing cross section}: Value of the absorbing cross
  section.

\item {\it Scattering cross section}: Value of the scattering cross
  section obtained by = extinction cross section- absorbing cross
  section.

\item {\it Scattering cross section with integration}: Value of the
  scattering cross section obtained by integration of the far field
  field radiated by the object.

\item {\it Scattering asymmetric parameter}: Asymmetric factor.

%\item {\it Optical force $x$}: Optical force according to the axis  $x$.

%\item {\it Optical force $y$}: Optical force according to the axis  $y$.

%\item {\it Optical force $z$}: Optical force according to the axis  $z$.

%\item {\it Optical force modulus}: Modulus of the optical force.

%\item {\it Optical torque $x$}:  Optical torque according to the axis  $x$.

%\item {\it Optical  torque $y$}: Optical torque according to the axis  $x$.

%\item {\it Optical  torque $z$}: Optical torque according to the axis  $x$.

%\item {\it Optical torque modulus} Modulus of the optical torque.

\end{itemize}

\section{Graphics}

\subsection{Plot epsilon/dipoles}

The button {\it Plot epsilon/dipoles} enables us to see the position
of each element of discretization. The colour of each point is
associated with the value of the permittivity of the considered
meshsize.


\subsection{Far field and microscopy}

\subsubsection{Plot Poynting vector}

{\it Plot Poynting}: enables us to draw the modulus of the Poynting
vector in 3D. If the option quick computation is taken, then the
results come from a interpolation of the points in the $(k_x,k_y)$
space: if it is not smooth increase the number of point of the FFT.

\subsubsection{Plot microscopy}

{\it Plot microscopy} : enables us to draw the diffracted field in the
Fourier plane for holographic microscope or the image through the dark
field, brightfield, phase or holographic microscope. We can plot the
modulus, intensity or the component $x$, $y$ or $z$.


We have 8 different plot:

\begin{itemize}

\item {\it Fourier plane: Scattered field: $k_z>0$} The diffracted
  field by the object in the Fourier plane for a holographic
  microscope in transmission.

\item {\it Fourier plane: Total field: $k_z>0$} The diffracted field
  by the object plus the incident field (if the incident field is a
  plane wave, then we have a Dirac) in the Fourier plane for a
  holographic microscope in transmission.

\item {\it Fourier plane: Scattered field: $k_z<0$} The diffracted
  field by the object in the Fourier plane for a holographic
  microscope in reflection.

\item {\it Fourier plane: Total field: $k_z<0$} The diffracted field
  by the object plus the incident field reflected on the multilayer
  system (if the incident field is a plane wave, then we have a Dirac)
  in the Fourier plane for a holographic microscope in reflection.

\item {\it Image plane: Scattered field: $z>0$} Image of the
  diffracted field by the object through the microscope in
  transmission.

\item {\it Image plane: Total field: $z>0$} Image of the diffracted
  field by the object plus the incident field through the microscope
  in transmission.
  

\item {\it Image plane: Scattered field: $z<0$} Image of the
  diffracted field by the object through the microscope in
  reflection.

\item {\it Image plane: Total field: $z<0$} Image of the diffracted
  field by the object plus the incident field through the microscope
  in reflection.
  

  
\end{itemize}


{\underline{Remarks}}


\begin{itemize}


\item The diffracted field is represented upon a regular mesh in
  $\Delta k_x=\Delta k_y$ such as $\sqrt{k_x^2+k_y^2} \le k_0$ NA. If
  the computation is done by radiation of the dipoles, then the code
  choose to have at least 21 points in the numerical aperture.  If
  quick computation is chosen then he code used FFT transform, then,
  the size of the picture is fixed by discretization of the object $d$
  with the relation $d \Delta k=2\pi/N$ and $N$ the size of the FFT.

  \item The computation in the image plane is always done with FFT.


\item For the brightfield, darkfield and phase microscope only the
  field in the image plane can be ploted. When the option $x$, $y$ and
  $z$ component is chosen, the phase can not be plotted as we sum
  incoherently all the incidences, then we get only the modulus.


\end{itemize}


\subsection{Study of the near field}

\begin{itemize}

\item The first button {\it Field} enables us to choose to represent
  the incident field, local field or macroscopic field.

\item The button {\it Type} enables us to represent the modulus or the 
component $x$, $y$ or $z$ of the studied field.

\item The button {\it Cross section $x$} ($y$ or $z$) enables us to
  choose the abscissa of the cut (ordinate or dimension). {\it Plot
    $x$} ($y$ or $z$) draws the cut in plane $x$. {\it Plot all $x$}
  draws all the cut at once.

\end{itemize}

%\subsection{optical force and torque}

%\begin{itemize}

%\item The first button {\it Field} enables us to choose to represent
%  the optical force or the optical torque.

%\item The button {\it Type} enables us to choose to represent the
%  modulus or the component $x$, $y$ or $z$ of the studied field.

%\item The button {\it Cross section $x$} ($y$ or $z$) enables us to
 % choose the abscissa of the cut (ordinate or azimuth). {\it Plot $x$}
 % ($y$ or $z$) draws the cur. {\it Plot all $x$} draws all the cuts at
 % once.

%\end{itemize}
