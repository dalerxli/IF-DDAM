\chapter{Ouput files for matlab, octave,
  scilab,...}\label{chap7}
\markboth{\uppercase{Ouput files for matlab, octave,
    scilab,...}}{\uppercase{Ouput files for matlab, octave,
    scilab,...}}

\minitoc

\section{Introduction}

It is not necessary to use the graphic interface of the program to
watch the results. For the scalar, all the results are in the output
file and for the pictures, it is possible to use directly the exit
files in ascii or in one hdf5 file and to read them through other
software such as Matlab, Octave, Scilab,...For example in the
directory bin the field ifdda.m uses matlab to represent the different
data.


When the advanced option is chosen, it is possible to choose to save
the data either in separate .mat files or in a single hdf5 file.

\begin{itemize}

\item In the case of the hdf5 file, there are six created groups:
  option (the options chosen by the user), near field (the near field
  data), microscopy (data from the microscopy), far field (data from
  the far field option), and dipole (position of the elements of
  discretization and permittivity).

\item In the case of .mat file, all the output are formatted in the
  form of a unique column vector or two column vectors if the number
  is a complex (the real part being associated with the first column
  and the imaginary part with the second column).

\item In the hdf5 file all the data are formatted under the form of a
  single column vector and with two separate tables in the case of
  complex numbers.

\item In the case where the file contains three-dimensional data,
  these ones are always stored as follows:

\hspace{5mm} {\bf do} i=1,nz

  \hspace{10mm} {\bf do} j=1,ny

 \hspace{15mm} {\bf do} k=1,nx

      \hspace{20mm}  write(*,*) data(i,j,k)
         
   \hspace{15mm} {\bf enddo}

  \hspace{10mm} {\bf enddo}

 \hspace{5mm} {\bf enddo}

 Three-dimensional data are going to be recognized by 3D at the
 beginning of the line.
\end{itemize}

\section{List of all exit files}

All the files have the extension .mat if you choose the output in
ascii file. 

\begin{itemize}
\item x,y,z represent the different used coordinates.
\item (3D) epsilon contain the permittivity of the object.
\item (3D) xc,yc,zc contain the coordinates of all the points of the
  mesh.
\item (3D) xwf,ywf,zwf contain the coordinates of all the points of
  the mesh in which the near field is calculated when the wide field
  option (wide field) is used.
\item (3D complex) incidentfieldx (y,z) contains the component x(y,z)
  of the incidental field only inside the object.
\item (3D) incidentfield contains the modulus of the incident field
  only inside the object.
\item (3D complexe) macroscopicfieldx (y,z) contains the component
  x(y,z) of the macroscopic field only inside the object.
\item (3D) contains the modulus of the macroscopic field only inside
  the object.
\item (3D complexe) localfieldx (y,z) contains the component x(y,z)
  of the local field only inside the object.
\item (3D) localfield contains the modulus of the local field only
  inside the object.
\item (3D complexe) incidentfieldxwf (y,z) contains the component
  x(y,z) of the incident field inside the box of near field in wide
  field.
\item (3D) incidentfieldwf contains the modulus of the incidental
  field inside the box of near field in wide field.
\item (3D complexe) macroscopicfieldxwf (y,z) contains the component
  x (y,z) of the macroscopic field inside the box of near field in wide
  field.
\item (3D) macroscopicfieldwf contains the modulus of the macroscopic
  field inside the box of near field in wide field.
\item (3D complexe) localfieldxwf (y,z) contains the component x(y,z)
  of the local field inside the box of near field in wide field.
\item (3D) localfieldwf contains the modulus of the local field inside
  the box of near field in wide field.
\item theta is a board which contains all the theta angles
  corresponding to all the directions in which the vector of Poynting
  is calculated. Its size is (Ntheta+1)*Nphi.
\item phi is a board which contains all the theta angles corresponding
  to all the directions in which the vector of Poynting is
  calculated. Its size is (Ntheta+1)*Nphi.
\item (2D) poynting contains the modulus of the vector of
  Poynting in theta and phi direction of size (Ntheta+1)*Nphi.
%\item (3D) forcex (y,z) contains the  $x$ component of the optical force
%  only inside the object.
%\item (3D) torquex (y,z) contains the $x$ component of the optical
%  torque force only inside the object.
\item (2D) poyntingneg and poyntingpos contain the Poynting modulus for
  $k_z<0$ and $k_z>0$ respectively in the plane $(k_x,k_y)$.
\item kx and ky contain the coordinates for poyntingneg and
  poyntingpos.
\item (2D) fourierpos(x,y,z) contains the diffracted field in the fourier
  plane in modulus (x,y,z) for $k_z>0$.
\item (2D) fourierposinc(x,y,z) contains the total field in the fourier
  plane in modulus (x,y,z) for $k_z>0$.
\item (2D) fourierneg(x,y,z) contains the diffracted field in the fourier
  plane in modulus (x,y,z) for $k_z<0$.
\item (2D) fourierneginc(x,y,z) contains the total field in the fourier
  plane in modulus (x,y,z) for $k_z<0$.
\item kxfourier (kyfourier) contains the abscissa (ordinate) of the
  Fourier plane.
\item (2D) imagepos(x,y,z) contains the diffracted field in the image
  plane in modulus (x,y,z) for $z>0$.
\item (2D) imageposinc(x,y,z) contains the total field in the image plane
  in modulus (x,y,z) for $z>0$.
\item (2D) imageneg(x,y,z) contains the diffracted field in the image
  plane in modulus (x,y,z) for $z<0$.
\item (2D) imageneginc(x,y,z) contains the total field in the image plane
  in modulus (x,y,z) for $z<0$.
\item (2D) imagebfpos(x,y,z) contains the diffracted field in the
  image plane in modulus (x,y,z) for a dark field with illumination
  inside the numerical aperture of the condenser for $z>0$.
\item (2D) imageincbfpos(x,y,z) contains the total field in the image
  plane in modulus (x,y,z) for a brightfield microscope for $z>0$.
\item (2D) imagebfneg(x,y,z) contains the diffracted field in the
  image plane in modulus (x,y,z) for a dark field with illumination
  inside the numerical aperture of the condenser for $z<0$.
\item (2D) imageincbfneg(x,y,z) contains the total field in the image
  plane in modulus (x,y,z) for a brightfield microscope for $z<0$.
\item (2D) imagedfpos(x,y,z) contains the diffracted field in the
  image plane in modulus (x,y,z) for a dark field with illumination
  with a hollow cone of light with a numerical aperture equal to the
  condenser lens for $z>0$.
\item (2D) imageincdfpos(x,y,z)contains field in the image plane in
  modulus (x,y,z) for a phase microscope for $z>0$.
\item (2D) imagedfneg(x,y,z) contains the diffracted field in the
  image plane in modulus (x,y,z) for a dark field with illumination
  with a hollow cone of light with a numerical aperture equal to the
  condenser lens for $z<0$.
\item (2D) imageincdfneg(x,y,z)contains field in the image plane in
  modulus (x,y,z) for a phase microscope for $z<0$.
\item (2D) Ediffkposx (y,z) contains the diffracted field by the
  object for $k_z>0$ in the $(k_x,k_y)$ plane.
\item (2D) Ediffknegx (y,z) contains the diffracted field by the
  object for $k_z<0$ in the $(k_x,k_y)$ plane.
\item ximage contains the $x$ position for the images of microscopy.
\item yimage contains the $y$ position for the images of microscopy. 
\item kxincident(bf) and (df) plots the position of the component $x$
  of the incident wave vector for the bf (df) microscope.
\item kyincident(bf) and (df) plots the position of the component $y$
  of the incident wave vector for the bf (df) microscope.

  
\end{itemize}

