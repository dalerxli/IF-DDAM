\chapter{Managing of the configurations}\label{chap2}
\markboth{\uppercase{Managing of the configurations}}{\uppercase{Managing
 of the configurations}}

\minitoc

\section{Introduction}


The Code is launched by ./cdm inside the bin folder for a Linux
configuration. It has been created to be as convenient as possible and
so needing few explanations for its use. However, certain conventions
have been taken and need to be clarified.

\section{Creation and saving of a new configuration}

In order to start a new calculation, go to the tab {\it calculation}
and {\it New}. A new configuration shows up with values by default.
Once the new configuration is chosen, in order to be saved, the tab
{\it Calculation} and {\it Save} have to be selected again.  Then, we
select the name of the configuration, and we may add a short
description of the calculation that has been made.  Another way to
save a configuration is to click directly on the panel of the
configuration {\it Save configuration}. Then, two fields appear, one
for the name of the configuration and the second one for its
description.

\section{Managing of the configurations}

In order to manage all the selected configurations, we have go to the
tab {\it Calculation} and {\it Load}. So, a new window appears with
all the saved configurations. For each configuration there is a short
description that the user has entered, the date, when the
configuration file has been saved, then the principal characteristics
of the configuration (wave length, power, the beam's waist, object,
material, discretization and tolerance of the iterative method). It is
enough to click on a configuration and to click on {\it load} in order
to load a configuration.

The {\it delete} button is used to delete a saved configuration and
the {\it export} enables to export inside a file (name of the
configuration.opt) all the characteristics of the configuration.

Note that by double clicking on the line, we can modify the
description field.


