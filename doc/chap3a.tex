\chapter{Properties of the illumination }\label{chap3}
\markboth{\uppercase{Properties of the illumination}}{\uppercase
{Properties of the illumination}}

\minitoc

\section{Introduction}

In the section properties of the illumination, the field {\it
  Wavelength} enables us to enter the using wavelength in vacuum. This
one is entered in nanometer. The field $P_0$ enables to enter the
power of the laser beam in Watt. The field $W_0$ in nanometer enables
to enter for a plane wave the radius of the laser beam and for a
Gaussian beam, the waist of the beam.

Note that the beam always propagates in the direction of the positive
$z$ axis, hence for $k_z>0$ whatever the beam chosen.

\section{Beam}

\subsection{Introduction}

There are six beams predefined, their propagation direction is always
defined in the same way, with two angles $\theta$ and $\varphi$,
except for he speckle.  They are connected to the given direction by
the wave vector as follows:
%%%%%%%%%%%%%%%%%%%%%%%%%%%%%%%%%%%%%%%%%%%%%
\be k_x & = & k_0 \sin \theta \cos\varphi \\
k_y & = & k_0 \sin \theta \sin\varphi \\
k_z & = & k_0 \cos \theta \ee
%%%%%%%%%%%%%%%%%%%%%%%%%%%%%%%%%%%%%%%%%%%%%
where $\ve{k}_0=(k_x,k_y,k_z)$ is the wave vector parallel to the direction
of the incident beam and $k_0$ the wave number, see
Fig.~\ref{faisceau}.
%%%%%%%%%%%%%%%%%%%%%%%%%%%%%%%%%%%%%%%%%%%%%%%
\begin{figure}[h]
\begin{center}
  \includegraphics*[width=8.0cm,draft=false]{faisceau.eps}
\end{center}
\caption{Definition of the beam's direction}
\label{faisceau}
\end{figure}
%%%%%%%%%%%%%%%%%%%%%%%%%%%%%%%%%%%%%%%%%%%%%%%
For the polarization, we use the plane $(x,y)$ as referential surface.
Then, we can determine a polarization TM ($p$) and TE ($s$) with the
presence of a surface, see Fig.~\ref{pola} or a polarization along he
$x$ or $y$ axis, depending of the beam.
%%%%%%%%%%%%%%%%%%%%%%%%%%%%%%%%%%%%%%%%%%%%%%%
\begin{figure}[h]
\begin{center}
  \includegraphics*[width=8.0cm,draft=false]{pola.eps}
\end{center}
\caption{Definition of the beam's polarization.}
\label{pola}
\end{figure}
%%%%%%%%%%%%%%%%%%%%%%%%%%%%%%%%%%%%%%%%%%%%%%%
The frame $(x,y,z)$ is used as an absolute referential.

\subsection{Linear plane wave }

{\it Linear plane wave} is a plane wave linearly polarized. The first
line is relative to $\theta$ and the second to $\varphi$. The third
line is connected to the polarization, polarization TM(1) or TE(0).
Note that the polarization is not necessarily purely in TE or TM:
${\rm pola}\in[0~1]$ such as $E^2_{\rm TM}={\rm pola}^2E^2$ and
$E^2_{\rm TE}=(1-{\rm pola}^2)E^2$. Notice that if one wants a
polarisation only along $x$ ($y$) direction, one can choose pola=2 (3)
and the code will compute the right value of pola to get the
polarisation asked.

Note that the phase is always taken null at the origin of the frame,
with ${\rm Irradiance}=P_0/S$ where $S=\pi w_0^2$ is the surface of
the beam and $E_0=\sqrt{2 {\rm Irradiance}/c/\varepsilon_0}$.


\subsection{Circular plane wave }

{\it pwavecircular } is a plane wave circularly polarized. The first
line is relative to $\theta$ and the second to $\varphi$. The third
line is connected to the polarization that we can choose right (1) or
left (-1) circular.

Note that the phase is taken null at the origin of the frame, with
${\rm Irradiance}=P_0/S$ where $S=\pi w_0^2$ is the surface of the
beam and $E_0=\sqrt{2 {\rm Irradiance}/c/\varepsilon_0}$.


\subsection{Multiple plane wave}

{\it Multiple wave} consists to take many planes waves. The first
thing to do is to choose the number of plane wave, and then for each
plane wave we choose $\theta$ and $\varphi$ and the polarization. We
have to write also the complex magnitude of each plane wave. The sum
of the power of all the plane wave is equal to $P_0$.


\subsection{Linear Gaussian beam}


{\it Linear Gaussian} is a Gaussian wave polarized linearly. The first
line is relative to $\theta$ and the second to $\varphi$. The third
line is connected to the angle betwwen the polarization and the $x$
axis (0) or along the $y$ axis (90).

The three following lines help to fix the position of the centre
$(x_0,y_0,z_0)$ of the waist in nanometers in the frame $(x,y,z)$.

Note that this Gaussian beam may have a very weak waist, because it is
calculated without any approximation through an angular spectrum
representation done with FFT with always $k_z>0$ whatever the
inclination $\theta$ of the Beam.  The definition of the waist, for a
beam propagating along the $z$ axis is :\cite{Agrawal_JOSA_79}
%%%%%%%%%%%%%%%%%%%%%%%%%%%%%
\be E(x,y,0)= E_0 e^{-\rho^2/(2 w_0^2)}, \ee
%%%%%%%%%%%%%%%%%%%%%%%%%%%%%
with $\rho=\sqrt{x^2+y^2}$. Then to compute a Gaussian beam polarized
along the $x$ axis, we have the magnitude of the Fourier component as:
%%%%%%%%%%%%%%%%%%%%%%%%%%%%%
\be \ve{A}(k_x,k_y)= E_0 (k_z\ve{i}-k_x\ve{k} )
\frac{1}{\sqrt{k_x^2+k_z^2}} w_0 e^{-(k_x^2+k_y^2)w_0^2/2}, \ee
%%%%%%%%%%%%%%%%%%%%%%%%%%%%%
then we compute the reference field, $\ve{A}_{\rm ref} (k_x,k_y,z)$,
through the multilayer system, and the reference field reads:
%%%%%%%%%%%%%%%%%%%%%%%%%%%%%
\be \ve{E}_{\rm ref}(x,y,z)= \int \int_{k_0} \ve{A}_{\rm ref}
(k_x,k_y,z) e^{i(k_x (x-x_0)+k_y (y-y_0)-k_zz_0 )} {\rm d}
\ve{k}_{\parallel}. \ee
%%%%%%%%%%%%%%%%%%%%%%%%%%%%%


\subsection{Circular Gaussian}

{\it Circular Gaussian} is a Gaussian wave circularly polarized. The
first line is relative to $\theta$ and the second to $\varphi$. The
third line is connected to the polarization that we can choose right
(1) or left (-1) circular.


The next three lines enable us to fix the position of the centre of the waist
in nanometers in the frame $(x,y,z)$.

Note that this Gaussian wave may have a very weak waist, because it is 
calculated without any approximation through a plane wave spectrum.


\subsection{Speckle}

{\it Speckle} is done as the Gaussian beam with a FFT but with a
magnitude with a random phase. For a speckle polarized along the $x$
axis  the magnitude of the Fourier component is:
%%%%%%%%%%%%%%%%%%%%%%%%%%%%%
\be \ve{A}(k_x,k_y)= E_0 (k_z\ve{i}-k_x\ve{k} )
\frac{1}{\sqrt{k_x^2+k_z^2}} e^{i \varphi} , \ee
%%%%%%%%%%%%%%%%%%%%%%%%%%%%%
where $\varphi$ is random variable between 0 and $2\pi$. Then we
compute the reference field, $\ve{A}_{\rm ref} (k_x,k_y,z)$, through
the multilayer system, and the reference field reads:
%%%%%%%%%%%%%%%%%%%%%%%%%%%%%
\be \ve{E}_{\rm ref}(x,y,z)=\int \int_{k_0 {\rm NA}} \ve{A}_{\rm ref}
(k_x,k_y,z) e^{i(k_x (x-x_0)+k_y (y-y_0)-k_zz_0 )} {\rm d} \ve{k}_{\parallel}, \ee
%%%%%%%%%%%%%%%%%%%%%%%%%%%%%
where ${\rm NA}$ is the numerical aperture of the
microscope. $\ve{r}_0$ permits to shift the speckle and the seed to
change the distribution of the speckle.



\subsection{Arbitrary wave}

In the case of an arbitrary field, the characteristic are determined
by the user.  In other words, he has to create the field himself, and
it is mandatory to create these files respecting the chosen
conventions by the code.


The description of the discretization of the incident field is done 
within a file which is asked for when we click on {\it Props}.
For example, for the real part of the component $x$ of the field, 
it has to be constructed as follows:

nx,ny,nz 

dx,dy,dz

xmin,ymin,zmin

\begin{itemize}
\item  nx is the number of meshsize according to the axis $x$
\item  ny is the number of meshsize according to the axis $y$
\item  nz is the number of meshsize according to the axis $z$
\item  dx is the step according to the axis $x$
\item  dy is the step according to the axis $y$
\item  dz is the step according to the axis $z$
\item xmin the smallest abscissa
\item ymin the smallest ordinate
\item zmin the smallest azimuth
\end{itemize}

Then, the files of the electric field are created as follows for 
each of the components of the real part and separated imaginary field:

\vspace{10mm}

open(11, file='Exr.mat', status='new', form='formatted', access='direct', recl=22)

{\bf do} k=1,nz

\hspace{5mm} {\bf do} j=1,ny

\hspace{10mm} {\bf do} i=1,nx 

\hspace{15mm} ii=i+nx*(j-1)+nx*ny*(k-1)

\hspace{15mm} write(11,FMT='(D22.15)',rec=ii) dreal(Ex)

\hspace{10mm} {\bf enddo}

\hspace{5mm} {\bf enddo}

{\bf enddo}

\vspace{10mm}

Be careful, the mesh size of the discretization of the object has to
be larger than the meshsize of the discretization of the field.
