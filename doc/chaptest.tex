\chapter{Fichiers de test}\label{chaptest}
\markboth{\uppercase{Fichiers de test}}{\uppercase{Fichiers de test}}

\minitoc

\section{Introduction}

Dans bin/tests est dispos� un fichier options.db3. Si on le copie un
directory en dessous ``cp options.db3 ../.'', quand on lance le code
apr�s un load il appara�t quatre configurations test qui permettent de
voir toutes les options en action.

\section{Test1}

Le but du test1 est de tester un cas simple mais avec une pr�cision
importante et de nombreuses options du code afin de les valider. La
Fig.~\ref{test1conf} montre la configuration choisie.


%%%%%%%%%%%%%%%%%%%%%%%%%%%%%%%%%%%%%%%%%%%%%%%
\begin{figure}[h]
\begin{center}
  \includegraphics*[width=15.0cm,draft=false]{test1conf.eps}
\end{center}
\caption{Test1: configuration choisie.}
\label{test1conf}
\end{figure}
%%%%%%%%%%%%%%%%%%%%%%%%%%%%%%%%%%%%%%%%%%%%%%%

La Fig.~\ref{test1res} montre les r�sultats obtenus. Les trac�s sont
effectu�s avec Matlab, mais peuvent bien s�r �tre r�alis�s avec
l'interface graphique int�gr�.
%%%%%%%%%%%%%%%%%%%%%%%%%%%%%%%%%%%%%%%%%%%%%%%
\begin{figure}[h]
\begin{center}
  \includegraphics*[width=15.0cm,draft=false]{test1conf.eps}
\end{center}
\caption{Test1: r�sultats.}
\label{test1res}
\end{figure}
%%%%%%%%%%%%%%%%%%%%%%%%%%%%%%%%%%%%%%%%%%%%%%%



\section{test2}

\section{test3}

\section{test4}


\end{itemize}
