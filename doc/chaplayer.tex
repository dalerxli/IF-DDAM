\chapter{Propri�t�s du multicouche}\label{chapl}
\markboth{\uppercase{Propri�t�s du multicouche}}{\uppercase
{Propri�t�s du multicouche}}

\minitoc

\section{Introduction}


Dans la section propri�t�s du multicouche, vous devez d'abord choisir
le nombre d'interface. A noter que le chiffre z�ro est exclu, car cela
correspond au code d'espace homog�ne beaucoup plus efficace pour cette
configuration. Une fois le nombre $n$ d'interface choisi, cliquer sur
{\it Props}. Il appara�t une nouvelle fen�tre avec $n$ interfaces et
$n+1$ milieux. Le premier milieu correspond au substrat c'est � dire
le milieu par lequel arrive la lumi�re, forc�ment celui-ci est
transparent. Apr�s il faut rentrer la position de l'interface en
nanom�tre puis rentrer le milieu de la seconde interface et ainsi de
suite, pour arriver au dernier milieu (le superstrat) qui peut �tre
transparent ou absorbant.


Un exemple est donn� Fig.~\ref{configlayer}, on voit de suite qu'il
faut faire attention car la premi�re ligne correspond donc � la
permittivit� qui a le $z$ le plus faible. 
%%%%%%%%%%%%%%%%%%%%%%%%%%%%%%%%%%%%%%%%%%%%%%%
\begin{figure}
\begin{center}
  \includegraphics*[width=15.0cm,draft=false]{configlayer.eps}
\end{center}
\caption{Comment configurer le multicouche.}
\label{configlayer}
\end{figure}
%%%%%%%%%%%%%%%%%%%%%%%%%%%%%%%%%%%%%%%%%%%%%%%



\section{Remarques}

\begin{itemize}

\item Si on �tudie le microscope en transmission alors le dernier
  milieu doit �tre transparent.

\item Si l'objet �tudi� est � cheval sur une ou plusieurs interfaces,
  il faut savoir que les dip�les constituant l'objet ne peuvent pas
  �tre situ�s sur une interface. Si c'est le cas alors, le code bouge
  l�g�rement l'interface (moins d'une demie maille) pour que cela
  n'arrive pas.


\item Le code est limit� � 10 interfaces maximum.

\item Attention si il y a une grande distance entre deux interfaces
  (plusieurs dizaines ou centaines de longueur d'onde), il peut arriver
  que le code n'arrive pas � calculer le tenseur de Green.

 

  \item Le syst�me multicouche prend en compte les modes guid�s car le
    tenseur de Green est calcul� avec le th�or�me des r�disus.

  \end{itemize}
