\chapter{Possible study with the code}\label{chap5}
\markboth{\uppercase{Possible study with the
    code}}{\uppercase{Possible study with the code}}

\minitoc

\section{Introduction}

To determine the object with the appropriate orientation is not an
easy task.  That is why the first option {\it Only dipoles with
  epsilon}, enables us to check quickly if the object entered is well
the one intended without any calculation being launched. Once this has
been done, there are three great fields: the study in far field, the
study in near field and the optical forces.

\vskip10mm

{\underline{Important}}: Note that in the DDA the computation that
takes the longest time is the calculation of the local field due to
the necessity to solve the system of linear equations.  One option has
been added which consists in reading again the local field starting
with a file. When this option is selected, the name of a file is asked
for; either we enter an old file or a new name:

\begin{itemize}
\item If this is a new name, the calculation of the local field is
  going to be accomplished, then, stored together with the chosen
  configuration.
\item If this is an old name, the local field is going to be read
  again with a checking that the configuration has not been changed
  between the writing and the second reading. This makes it easier to
  relaunch calculations very quickly for the same configuration but
  for different studies.
\end{itemize}

\section{Study in far field}

When the option far field is selected, three possibilities appear:

\begin{itemize}

\item {\it Cross section}: This option enables us to calculate the
  extinction ($C_{\rm ext}$), absorbing ($C_{\rm abs}$) and scattering
  cross section ($C_{\rm sca}$). The scattering cross section is
  obtained through $C_{\rm sca}=C_{\rm ext}-C_{\rm abs}$. The
  extinction and absorption cross sections may be evaluated as:
%%%%%%%%%%%%%%%%%%%%%%%%%%%%%%%%%%%%%%%%%%%%%%%%
  \be C_{\rm ext} & = & \frac{4\pi k_0}{\|\ve{E}_0\|^2} \sum_{j=1}^{N}
  {\rm Im} \left[ \ve{E}^*_0(\ve{r}_j).  \ve{p}(\ve{r}_j) \right] \\
  C_{\rm abs} & = & \frac{4\pi k_0}{\|\ve{E}_0\|^2} \sum_{j=1}^{N}
  \left[ {\rm Im} \left[ \ve{p}(\ve{r}_j). (\alpha^{-1}(\ve{r}_j))^*
      \ve{p}^*(\ve{r}_j) \right] -\frac{2}{3} k_0^3
    \| \ve{p}^*(\ve{r}_j) \|^2 \right] .\ee
%%%%%%%%%%%%%%%%%%%%%%%%%%%%%%%%%%%%%%%%%%%%%%%%
  Note that the cross section can only be computed if the background
  is homogeneous. The free space code is more adapted in that case,
  but it permits to check easily if the discretization is well adapted
  when a sphere is studied. Notice that the cross section can be
  computed if there is no interface.



  
\item {\it Cross section+Poynting}: This option calculates also the
  scattering cross section from the integration of the far field
  diffracted by the object upon 4$\pi$ steradians, the asymmetric
  factor when the background is homogeneous, else it computes only the
  differential cross section, {\it i.e.}
  $\left< {\cal P} \right>=\left< \ve{S} \right> .\ve{n} R^2$ with
  $\ve{S}$ the Poynting vector, $\ve{n}$ the direction of observation,
  which is going to be represented in 3D. The values {\it Ntheta} and
  {\it Nphi} enable us to give the number of points used in order to
  calculate the scattering cross and to represent the Poynting
  vector. The larger the object is, the larger {\it Ntheta} and {\it
    Nphi} must be, which leads to time consuming calculations for
  objects of several wavelengths.
%%%%%%%%%%%%%%%%%%%%%%%%%%%%%%%%%%%%%%%%%%%%%%%%
  \be C_{\rm sca} & = & \frac{k_0^4}{\|\ve{E}_0\|^2} \int \left\|
    \sum_{j=1}^N \left[ \ve{p}(\ve{r}_j)-\ve{n}(\ve{n}.
      \ve{p}(\ve{r}_j)) \right] e^{-i k_0 \ve{n}.\ve{r}_j} \right\|^2
  {\rm d}\Omega \\ g & = & \frac{k_0^3}{C_{\rm sca} \|\ve{E}_0\|^2}
  \int \ve{n}.\ve{k}_0 \left\| \sum_{j=1}^N \left[
      \ve{p}(\ve{r}_j)-\ve{n}(\ve{n}.  \ve{p}(\ve{r}_j)) \right] e^{-i
      k_0 \ve{n}.\ve{r}_j} \right\|^2 {\rm d}\Omega \\
  \frac{{\rm d} \left< {\cal P} \right>}{{\rm d}\Omega} & = & \frac
  {1}{2} c \varepsilon_0 n \left\| \ve{E}_{\rm d} (\ve{k}_{\parallel})
  \right\|^2,  \ee
%%%%%%%%%%%%%%%%%%%%%%%%%%%%%%%%%%%%%%%%%%%%%%%%
  where $ \ve{E}_{\rm d} (\ve{k}_{\parallel})$ is the diffracted field
  in far field.

  A solution in order to go faster (option {\it quick computation})
  and to pass by FFT for the calculation of the diffracted field.  In
  this case, of course, it is convenient to discretize keeping in mind
  that the relation $\Delta x \Delta k=2\pi/N$ connects the mesh size
  of the discretization with the size of the FFT.  This is convenient
  for objects larger than the wavelength. Indeed, $L=N\Delta x$
  corresponds to the size of the object which gives $\Delta k=2\pi/L$,
  and if the size of the object is too small, then, the $\Delta k$ is
  too large, and the quadrature is imprecise. Note that since the
  integration is performed on two planes parallel to the plane
  $(x,y)$, is not convenient if the incident makes an angle more than
  70 degrees with the $z$ axis. The 3D representation of the vector of
  Poynting is done as previously, i.e. with {\it Ntheta} and {\it
    Nphi} starting with an interpolation upon the calculated points
  with the FFT.

\item {\it Emissivity}. This study computes the reflectance,
  transmittance and absorptance. If the object under study is no
  absorbing then the absorptance should be zero. Then it traduces the
  level of energy conservation of our solver. It can depend of the
  precision of the iterative method and of the polarizability chosen.

\end{itemize}
  
\section{Microscopy}
  
This option permits to compute the image obtained for different
microscope (holographic, brightfield, darkfield and phase). We
consider a microscope made of an objective lens and a tube lens in
$4f$ configuration and sine-Abbe condition~\cite{Abbe} It asks for the
numerical aperture of the objective lens
(${\rm NA}=n_{\rm obj} \sin\theta_{\rm obj}$) and the numerical
aperture of the condenser lens ( $sin \theta_{\rm cond}$) for
brightfield, darkfield and phase microscope.  By default, the lenses
are placed parallel to the plane $(x,y)$ and their optical axis are
confounded with the $z$ axis. The focus plane of the lenses are placed
to the origin of the frame but can be changed via the field ``Position
of the focal plane'' for the microscope in transmission and reflection
(Fig.~\ref{lentille}). The magnification of the microscope is $M$ and
should be above 1. The drop menu propose three different microscope.


\begin{itemize}

\item {\it Holographic}: This option computes the diffracted field
  (Fourier plane) with the incident field defined in the section
  illumination properties. It computes the image plane with or without
  the presence of the incident field.


  
\item {\it Brightfield}: This microscope uses a condenser lens, which
  focuses light from the light source onto the sample with a numerical
  aperture defined below the magnification. It consists to sum
  incoherently the image obtained with many incident field inside this
  numerical aperture with different polarization, hence it can take
  time as it needs to solve many direct problem. The result is given
  in the image plane without the incident field (a kind of dark field)
  and with the incident field (brightfield).

\item {\it Darkfield \& phase}: In darkfield microscopy the condenser
  is designed to form a hollow cone of light with a numerical aperture
  equal to the condenser lens, as apposed to brightfield microscopy
  that illuminates the sample with a full cone of light. The result is
  given in the image plane (scattered field). In the phase microscopy
  the ring-shaped illuminating light that passes the condenser annulus
  is focused on the specimen by the condenser exactly as in the dark
  field microscope and then the incident field with a phase shifted of
  $\pi/2$ is added to the scattered field.
  
\end{itemize}

The drop menu side computation permits to simulate microscope in
transmission (Side $k_Z>0$), in reflection (Side $k_Z>0$), or both
cases. 


\begin{figure}[h]
\begin{center}
\includegraphics*[draft=false,width=150mm]{microscopie.eps}
\caption{Simplified figure of the microscope. The object focus of the
  objective lens are at the origin of the frame but canbe changed. The
  axis of the lens is confounded with the $z$ axis.}
\label{lentille}
\end{center}
\end{figure}


The calculation for the diffracted field may be completed starting
with the sum of the radiation of the dipoles (very long when the
object has a lot of dipoles) or with FFT (option {\it quick
  computation}) with a value $N=128$ by default here as well. In this
case, $\Delta x \Delta k=2\pi/N$ with $\Delta x$ the mesh size of
discretization of the object which corresponds also to the
discretization of the picture plane. Consequently, this one has a size
of $L=G N \Delta x$.


The principle of the computation of the far field diffracted by the
object and how to get the image through a microscope with a magnifying
factor $M>1$ has been detailed in Ref~\cite{Khadir_JOSAA_19}. Then we
recall it briefly.  The diffracted field in far field at a distance
$r$ of the origin in the direction $(k_x,k_y)$ can be written as
$\ve{E}= \ve{S}(k_x,k_y) \frac{e^{i k r}}{r}$. The field after the
first lens (field in the Fourier space) is then defined as:
$\ve{e}(k_{\parallel})=\frac{\ve{S}(k_x,k_y)}{-2 i \pi \gamma}$ with
$\gamma=\sqrt{\epsilon_{\rm mul} k_0^2-k_x^2-k_y^2}$ where the value
of $\epsilon_{\rm mul}$ corresponds to the permittivity of the
substrate for the microscope in reflection and to the permittivity of
the superstrate for the microscope in transmission.  A microscope
transforms a plane wave with wavevector $\ve{k}$ into a plane wave
with wavevector $\ve{k}'$ with $\ve{k}'=[\ve{k}'_{\parallel},\gamma']$
where $\ve{k}'_{\parallel}=(-k_x/M,-k_y/M)$ and
$\gamma'=\sqrt{k_0^2-k_{\parallel}^{'2}}$ (the refractive index in the
image space is considered equal to 1). Then the field in the image
space, after the tube lens, reads as:
%%%%%%%%%%%%%%%%%%%%%%%%%%%%%%%%%%%%%%%%%%%%%%%%%%%%%%%%%%%%%%%%%
\be \ve{E}_\mathrm{ob}(\ve{r})=\frac{1}{M} \iint
\sqrt{\frac{\gamma}{\gamma'}} \tilde{h}(\ve{k}_{\parallel})
\ve{e}'(\ve{k}_{\parallel}) \exp [i \ve{k}' \cdot (\ve{r} -\ve{r}_f) ]
   {\rm d} \ve{k}_{\parallel},
\label{objectfield}
\ee
%%%%%%%%%%%%%%%%%%%%%%%%%%%%%%%%%%%%%%%%%%%%%%%%%%%%%%%%%%%%%%%%%
where $\tilde{h}(\ve{k}_{\parallel})$ is cutoff function allowing the
transmission of only the signal included in the numerical aperture
($NA$) of the objective lens, it reads $\tilde{h}(\ve{k}_{\parallel})
= 1$ for $\mid \ve{k}_{\parallel} \mid < k_0 \mathrm{NA} $ and $0$
elsewhere and $\ve{r}_f$ the position of the lens. We have
%%%%%%%%%%%%%%%%%%%%%%%%%%%%%%%%%%%%%%%%%%%%%%%%%%%%%%%%%%%%%%%%%
\be \ve{e}'(\ve{k}_{\parallel}) =
\ve{R}(\ve{k}_{\parallel})\ve{e}(\ve{k}_{\parallel}),   \ee
%%%%%%%%%%%%%%%%%%%%%%%%%%%%%%%%%%%%%%%%%%%%%%%%%%%%%%%%%%%%%%%%%
and $\ve{R}(\ve{k}_\parallel)$ is given by:
%%%%%%%%%%%%%%%%%%%%%%%%%%%%%%%%%%%%%%%%%%%%%%%%%%%%%%%%%%%%%%%%%
\begin{equation}
\ve{R}(\ve{k}_\parallel)=
\begin{pmatrix}
  u_{x}^2(1-\cos\theta)+\cos\theta & u_{x} u_{y}(1-\cos\theta) & u_{y}
  \sin\theta \\ u_{x} u_{y}(1-\cos\theta) &
  u_{y}^2(1-\cos\theta)+\cos\theta & -u_{x} \sin\theta\\ -u_{y}
  \sin\theta & u_{x} \sin\theta & \cos\theta
\end{pmatrix},
\end{equation}
%%%%%%%%%%%%%%%%%%%%%%%%%%%%%%%%%%%%%%%%%%%%%%%%%%%%%%%%%%%%%%%%%
where $\ve{u}=\frac{\hat{\ve{k}}\times\ve{z}}{\mid \hat{\ve{k}} \times
  \ve{z} \mid}$ is the rotation axis. Notice that $\ve{u}$ has no
component along the $z$ direction. $\theta$ is defined as $\cos
\theta=\hat{\ve{k}}.\hat{\ve{k}'} $ and $\sin
\theta=\|\hat{\ve{k}}\times \hat{\ve{k}'}\|$


\section{Study in near field}

When the option near field is selected, two possibilities appear:

\begin{itemize}

\item {\it Local field}: This option enables us to draw the local
  field to the position of each element of discretization. The local
  field being the field at the position of each element of
  discretization in absence of itself. 

\item {\it Macroscopic field}: This option enables us to draw the
  macroscopic field to the position of each element of
  discretization. The connection between the local field and the
  macroscopic field is given Ref.~\cite{Chaumet_PRE_04} :
%%%%%%%%%%%%%%%%%%%%%%%%%%%%%%
  \be \ve{E}_{\rm macro} & = & 3 \varepsilon_{\rm mul} \left(
  \varepsilon+2\varepsilon_{\rm mul} -i \frac{k_0^3 d^3
    \varepsilon_{\rm mul}^{3/2}}{2 \pi} (\varepsilon-\varepsilon_{\rm
    mul})\right)^{-1} \ve{E}_{\rm local} \ee
%%%%%%%%%%%%%%%%%%%%%%%%%%%%%%


\end{itemize}

The last option enables us to choose the mesh in which the local and
macroscopic fields are represented.

\begin{itemize}

\item {\it Object}: Only the field in the object is
  represented. Notice that when FFT is used for the beam or for the
  computation of the diffracted field then this options is passed in
  the option {\it Cube}. This is same for the computation of the
  emissivity, the reread option and the use of the BPM(R).

\item {\it Cube}: The field is represented within a cuboid containing
  the object.

\item {\it Wide field}: The field is represented within a box greater
  than the object.  The size of the box correspond to the size of the
  object plus the Additional side band ($x$, $y$ or $z$) on each
  side. For example for a sphere with a radius $r=100$~nm and
  discretization of 10, {\it i.e.} a meshsize of 10 nm, with an
  Additional side band $x$ of 2, 3 for $y$ and 4 for $z$, we get a box
  of size:
%%%%%%%%%%%%%%%%%%%%%%%%%%%%%%%%%%%%%%%%%%
  \be l_x & = & 100 + 2\times 2 \times 10 = 140~{\rm nm} \\
  l_y & = & 100 + 2\times 3 \times 10 = 160~{\rm nm} \\
  l_z & = & 100 + 2\times 4 \times 10 = 180~{\rm nm}
  \ee
\end{itemize}
The field inside the wide field area in near field is computed with
%%%%%%%%%%%%%%%%%%%%%%%%%%%%%%%%%%%%%%%%%%
  \be \ve{E}=\ve{E}_0+\ve{A} \ve{D} \ve{E}, \ee
%%%%%%%%%%%%%%%%%%%%%%%%%%%%%%%%%%%%%%%%%%
which gives te field inside theh near field zone and in the object.


%\section{Optical force and torque}

%When the force option is selected, four possibilities appear:
%%\begin{itemize}

%\item {\it Optical force}: Calculation of the optical force exerting
%  on one or more objects.

%\item {\it Optical force density}: Enables us to draw the density of
%  the optical force.

%\item {\it Optical torque}: Calculation of the optical torque exerting
%  on one or more objects.  The torque is computed for an origin placed
%  in the gravity center of the object.

%\item {\it Optical torque density}: Enables us to draw the density of
%  the optical force torque.
%\end{itemize}
%The net optical force and troque experienced by the object are
%computed with~\cite{Chaumet_OL_00,Chaumet_JAP_07a}:
%%%%%%%%%%%%%%%%%%%%%%%%%%%%%%%%%%%%%%%%%%%%%%%%%
%\be \ve{F} & = & (1/2) \sum_{j=1}^N {\rm Re}\left(\sum_{v=1}^{3}
%  p_v(\ve{r}_j) \frac{\partial (E_v(\ve{r}_j))^*}{\partial u}\right) \\
%\ve{\Gamma} & = & \sum_{j=1}^N \left[ \ve{r}_{j} \times
%  \ve{F}(\ve{r}^g_{j})+ \frac{1}{2} {\rm Re} \left\{ \ve{p}(\ve{r}_{j})
%    \times \left[ \ve{p}(\ve{r}_{j})/{\alpha_{\rm
%          CM}}(\ve{r}_{j})\right]^* \right\} \right].  \ee
%%%%%%%%%%%%%%%%%%%%%%%%%%%%%%%%%%%%%%%%%%%%%%%%%
%where $u$ or $v$, stand for either $x$ ,$y$, or $z$. The symbol $*$
%denotes the complex conjugate. $\ve{r}^g_{j}$ is the vector bewteen
%$j$ and the center of masse of the object.
