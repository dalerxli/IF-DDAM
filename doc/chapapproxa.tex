\chapter{Approximation to increase the efficiency of the code}\label{chapapprox}
\markboth{\uppercase{Approximated method}}{\uppercase{Approximated method}}

\minitoc

\section{Introduction}

In the previous chapter we have presented the DDA in a simple way
where the object under study is a set of radiating dipole. In an
approach more rigorous, with the Maxwell's equation, we get in
Gaussian unit:
%%%%%%%%%%%%%%%%%%%%%%%%%%%%%%%%%%%%%%%%%%%%%%%%%%
\be \venab \times \ve{E}^{\rm m}(\ve{r}) & = & i \frac{\omega}{c}
\ve{B}(\ve{r}) \\
\venab \times \ve{B}(\ve{r}) & = & -i \frac{\omega}{c}
\varepsilon(\ve{r}) \ve{E}^{\rm m}(\ve{r}), \ee
%%%%%%%%%%%%%%%%%%%%%%%%%%%%%%%%%%%%%%%%%%%%%%%%%%
where $\varepsilon(\ve{r})$ denotes the relative permittivity of the
object and $\ve{E}^{\rm m}$ the macroscopic field inside the object,
then we get
%%%%%%%%%%%%%%%%%%%%%%%%%%%%%%%%%%%%%%%%%%%%%%%%%%
\be \venab \times ( \venab \times \ve{E}^{\rm m}(\ve{r}) ) & = &
\varepsilon(\ve{r}) k_0^2 \ve{E}^{\rm m}(\ve{r}), \ee 
%%%%%%%%%%%%%%%%%%%%%%%%%%%%%%%%%%%%%%%%%%%%%%%%%%
with $k_0=\omega^2/c^2$. Using the relationship
$\varepsilon=\varepsilon_{\rm mul}+4\pi \chi$, where $\chi$ denotes
the linear field susceptibility and $\varepsilon_{\rm mul}$ the
relative permittivity of the multilayer system. Note that
$\varepsilon_{\rm mul}$ depends only of $z$. Then, we have:
%%%%%%%%%%%%%%%%%%%%%%%%%%%%%%%%%%%%%%%%%%%%%%%%%
\be \venab \times ( \venab \times \ve{E}^{\rm m}(\ve{r}) )
-\varepsilon_{\rm mul} k_0^2 \ve{E}^{\rm m}(\ve{r}) & = & 4\pi
\chi(\ve{r}) k_0^2 \ve{E}^{\rm m}(\ve{r}) . \label{champref}\ee
%%%%%%%%%%%%%%%%%%%%%%%%%%%%%%%%%%%%%%%%%%%%%%%%%%
To solve this equation one needs the Green function defined as:
%%%%%%%%%%%%%%%%%%%%%%%%%%%%%%%%%%%%%%%%%%%%%%%%%%
\be \venab \times ( \venab \times \ve{G}(\ve{r},\ve{r}') )
-\varepsilon_{\rm mul} k_0^2 \ve{G}(\ve{r},\ve{r}') & = & 4\pi k_0^2
\ve{I} \delta(\ve{r}-\ve{r}'), \ee
%%%%%%%%%%%%%%%%%%%%%%%%%%%%%%%%%%%%%%%%%%%%%%%%%%
and the solution of Eq.~(\ref{champref}) reads:
%%%%%%%%%%%%%%%%%%%%%%%%%%%%%%%%%%%%%%%%%%%%%%%%%%
\be\ve{E}^{\rm m}(\ve{r}) = \ve{E}_{\rm ref}(\ve{r}) +\int_{\Omega}
\ve{G}(\ve{r},\ve{r}') \chi(\ve{r}') \ve{E}^{\rm m}(\ve{r}') {\rm d}
\ve{r}',\ee
%%%%%%%%%%%%%%%%%%%%%%%%%%%%%%%%%%%%%%%%%%%%%%%%%%
where $\ve{E}_{\rm ref}$ is the field in the absence of the object and
$\Omega$ the support of the object under study. When we solve
Eq.~(\ref{champref}) the field $\ve{E}^{\rm m}$ corresponds to the
macroscopic field inside the object.  To solve Eq.~(\ref{champref}) we
discretize the object in a set of $N$ subunits with a cubic meshsize
$d$, then the integral equation becomes the sum of $N$ integrals:
%%%%%%%%%%%%%%%%%%%%%%%%%%%%%%%%%%%%%%%%%%%%%%%%%%
\be\ve{E}^{\rm m}(\ve{r}_i) = \ve{E}_{\rm ref}(\ve{r}_i)
+\sum_{j=1}^{N} \int_{V_j} \ve{G}(\ve{r}_i,\ve{r}') \chi(\ve{r}')
\ve{E}^{\rm m}(\ve{r}') {\rm d} \ve{r}',\ee
%%%%%%%%%%%%%%%%%%%%%%%%%%%%%%%%%%%%%%%%%%%%%%%%%%
with $V_j=d^3$.  Assuming the field, the Green function and the
susceptibility constant over a subunit we get:
%%%%%%%%%%%%%%%%%%%%%%%%%%%%%%%%%%%%%%%%%%%%%%%%%%
\be\ve{E}^{\rm m}(\ve{r}_i) = \ve{E}_{\rm ref}(\ve{r}_i) +\sum_{j=1}^N
\ve{G}(\ve{r}_i,\ve{r}_j) \chi(\ve{r}_j) \ve{E}^{\rm m}(\ve{r}_j)
d^3.\ee
%%%%%%%%%%%%%%%%%%%%%%%%%%%%%%%%%%%%%%%%%%%%%%%%%%
We can share into two parts the Green function,
$\ve{G}=\ve{M}+\ve{T}$, where $\ve{M}$ is the Green function who take
into account of the multiple reflection between the different layers
and $\ve{T}$ is the Green function of the homogeneous space.  Using,
in first approximation (the radiative reaction term neglected)
$\int_{V_i}\ve{T}(\ve{r}_i,\ve{r}') {\rm d} \ve{r}'=
-4\pi/(3\varepsilon_{\rm mul}(\ve{r}_i))$~\cite{Yaghjian_PIEEE_80}, we
get:
%%%%%%%%%%%%%%%%%%%%%%%%%%%%%%%%%%%%%%%%%%%%%%%%%%
\be\ve{E}^{\rm m}(\ve{r}_i) = \ve{E}_{\rm ref}(\ve{r}_i) +\sum_{j=1}^N
\ve{G}'(\ve{r}_i,\ve{r}_j) \chi(\ve{r}_j) d^3 \ve{E}^{\rm
  m}(\ve{r}_j)-\frac{4\pi}{3\varepsilon_{\rm
    mul}(\ve{r}_i)}\chi(\ve{r}_i) \ve{E}^{\rm m}(\ve{r}_i),\ee
%%%%%%%%%%%%%%%%%%%%%%%%%%%%%%%%%%%%%%%%%%%%%%%%%%
where $\ve{G}'=\ve{M}+\ve{T}$ for $i \neq j$ and $\ve{G}'=\ve{M}$ for
$i=j$, then we can write
%%%%%%%%%%%%%%%%%%%%%%%%%%%%%%%%%%%%%%%%%%%%%%%%%%
\be\ve{E}(\ve{r}_i) & = & \ve{E}_{\rm ref}(\ve{r}_i) +\sum_{j=1}^N
\ve{G}'(\ve{r}_i,\ve{r}_j) \alpha_{\rm CM}(\ve{r}_j) \ve{E}(\ve{r}_j)
\\ {\rm with} \phantom{000} \ve{E}(\ve{r}_i) & = &
\frac{\varepsilon(\ve{r}_i)+2\varepsilon_{\rm
    mul}(\ve{r}_i)}{3\varepsilon_{\rm mul}(\ve{r}_i)} \ve{E}^{\rm
  m}(\ve{r}_i) \\ \alpha_{\rm CM}(\ve{r}_j) & = & \frac{3}{4\pi}
\varepsilon_{\rm mul}(\ve{r}_i) d^3
\frac{\varepsilon(\ve{r}_i)-\varepsilon_{\rm
    mul}(\ve{r}_i)}{\varepsilon(\ve{r}_i)+2\varepsilon_{\rm
    mul}(\ve{r}_i)} .\ee
%%%%%%%%%%%%%%%%%%%%%%%%%%%%%%%%%%%%%%%%%%%%%%%%%%
The field $\ve{E}(\ve{r}_i)$ is the local field, {\it i.e.}  the field
at the position $i$ in the absence of the subunit $i$. Then the linear
system can be written formally as
%%%%%%%%%%%%%%%%%%%%%%%%%%%%%%%%%%%%%%%%%%%%%%%%%%
\be \ve{E} =  \ve{E}_{\rm ref}+ \ve{A} \ve{D}_\alpha \ve{E}, \label{eqmsym}\ee
%%%%%%%%%%%%%%%%%%%%%%%%%%%%%%%%%%%%%%%%%%%%%%%%%%
where $\ve{A}$ is a matrix which contains all the Green function and
$\ve{D}_\alpha$ is a tridiagonal matrix with the polarizabilities of
each element of discretization. In the next chapter we detail how to
solve Eq.~(\ref{eqmsym}) rigorously, but in this present chapter we
detail different approached methods to avoid the tedious resolution of
Eq.~(\ref{eqmsym}).  The scattered field is computed through
%%%%%%%%%%%%%%%%%%%%%%%%%%%%%%%%%%%%%%%%%%%%%%%%%%
\be\ve{E}^{\rm d}(\ve{r}) & = & \sum_{j=1}^N \ve{G}(\ve{r},\ve{r}_j)
\alpha(\ve{r}_j) \ve{E}(\ve{r}_j). \ee
%%%%%%%%%%%%%%%%%%%%%%%%%%%%%%%%%%%%%%%%%%%%%%%%%%



\section{Approximated method}


\subsection{Born}


The most simple approximation is the Born approximation which consists
to assume the field inside the object equal to the reference field for
each element of discretization:
%%%%%%%%%%%%%%%%%%%%%%%%%%%%%%%%%%%%%%%%%%%%%%%%%%
\be \ve{E}^{\rm m}(\ve{r}_i) = \ve{E}_{\rm ref}(\ve{r}_i), \ee
%%%%%%%%%%%%%%%%%%%%%%%%%%%%%%%%%%%%%%%%%%%%%%%%%%
This approximation hold if the contrast is weak and the object small
compare to the wavelength of illumination.


\subsection{Renormalized Born }

The renormalized Born approximation consists to assume the local field
inside the object equal to the reference field :
%%%%%%%%%%%%%%%%%%%%%%%%%%%%%%%%%%%%%%%%%%%%%%%%%%
\be \ve{E}(\ve{r}_i) = \ve{E}_{\rm ref}(\ve{r}_i). \ee
%%%%%%%%%%%%%%%%%%%%%%%%%%%%%%%%%%%%%%%%%%%%%%%%%%
In that case the macroscopic field reads:
%%%%%%%%%%%%%%%%%%%%%%%%%%%%%%%%%%%%%%%%%%%%%%%%%%
\be \ve{E}^{\rm m}(\ve{r}_i) = \frac{3 \varepsilon_{\rm mul}
}{\varepsilon(\ve{r}_i)+2 \varepsilon_{\rm mul}} \ve{E}_{\rm
    ref}(\ve{r}_i). \ee
%%%%%%%%%%%%%%%%%%%%%%%%%%%%%%%%%%%%%%%%%%%%%%%%%%
This approximation is better that the classical Born approximation
when the permittivity is high.


\subsection{Born at the order 1}

To be more precise that the renormalized Born approximation, one can
perform the Born series at the order one:
%%%%%%%%%%%%%%%%%%%%%%%%%%%%%%%%%%%%%%%%%%%%%%%%%%
\be\ve{E}(\ve{r}_i) & = & \ve{E}_{\rm ref}(\ve{r}_i) +\sum_{j=1}^N
\ve{G}'(\ve{r}_i,\ve{r}_j) \alpha(\ve{r}_j) \ve{E}_{\rm
  ref}(\ve{r}_j). \ee
%%%%%%%%%%%%%%%%%%%%%%%%%%%%%%%%%%%%%%%%%%%%%%%%%%
In that case we take into account the simple scattering.

\section{Computation of the Green function}


Even though we are using an efficient integration scheme to evaluate
the Green tensor, see Ref.~\onlinecite{Paulus_PRE_00}, it takes a lot
of time to evaluate it for all the different pair of points covering
the object. Note that due to the translational invariance of the
reference medium in the $(x,y)$ plane,
$\ve{G}(\ve{r},\ve{r}')=\ve{G}(\|\ve{r}_{\parallel}-\ve{r}_{\parallel}'\|,z,z')$. For
each couple $(z,z')$, the number of pairs with different distances of
a Cartesian $(x,y)$ mesh with $n_x\times n_y$ points ($n_x>n_y)$ is
equal to $n_y(2 n_x-n_y+1)/2$. To accelerate the computation, we
approximate $\ve{G}(\|\ve{r}_{\parallel}-\ve{r}_{\parallel}'\|,z,z')$
using an interpolation of a discrete set of points,
$\ve{G}(qd/n_d,z,z')$ with $q=1,\cdots, {\rm int}\sqrt{ n_x^2+n_y^2}$
and $n_d$ a natural number. Linear and polynomial interpolations could
not evaluate the Green tensor properly when
$\|\ve{r}_{\parallel}-\ve{r}_{\parallel}'\|<\lambda$, as the fast
decay of the evanescent waves was not accounted for accurately. We
obtained much better results using rational functions, that is
quotients of polynomials. Rational functions have the ability to model
functions with poles (Press et al. 1986) and permit an accurate
approximation of the $1/r^3$ behavior of the Green tensor in the near
field range.

In the code, in the section numerical parameters, the drop menu Green
function permits to choose to compute rigorously the Green function
or evaluate the Green tensor with interpolation for $n_d=1, 2, 3, 4$.
Notice that by default the code uses $n_d=2$ which is enough
accurate. Note that only the computation with interpolation is
parallelized.
