\chapter{Properties of the multilayer }\label{chapl}
\markboth{\uppercase{Properties of the multilayer}}{\uppercase
{Properties of the multilayer}}

\minitoc

\section{Introduction}

In the section properties of the multilayer, you should first choose
the number of interface. Note that you can not choose zero, as in that
case the free space code is more efficient. Once the number $n$ of
interface is fixed click on {\it Props}. It appears a new window with
$n$ interfaces and $n+1$ media. The first medium corresponds to the
substrate, {\it i.e.} the medium through which the light comes. The
permittivity of this medium is obviously transparent (no absorption).
Then the position of the first interface in nanometer is asked
followed by the second medium, etc. The sign -is associated to the
substrat and the sign + to the superstrat, then $\varepsilon_{-}$ and
$\varepsilon_{+}$ for the permittivity, respectiveley.


An example is given in Fig.~\ref{configlayer}
%%%%%%%%%%%%%%%%%%%%%%%%%%%%%%%%%%%%%%%%%%%%%%%
\begin{figure}
\begin{center}
  \includegraphics*[width=15.0cm,draft=false]{configlayer.eps}
\end{center}
\caption{How to configure the layer.}
\label{configlayer}
\end{figure}
%%%%%%%%%%%%%%%%%%%%%%%%%%%%%%%%%%%%%%%%%%%%%%%


The last medium corresponds to the superstrate.


\section{Remarks}

\begin{itemize}

\item The last medium if one studies a microscope in transmission can
  not be absorbing.

\item If the object under study contains one or more interfaces, note
  that any dipole (element of discretization) can be on a interface,
  then the code slightly moves the interface (less than one half mesh)
  to avoid this.

\item The code is limited to 10 interfaces.

\item Notice that if there is a large distance between two layers
  (several hundred of wavelength) the computation of the Green function
  can fail.

  \item The multilayer support guided mode resonance as the Green
    function is computed with the residue theorem.

  \end{itemize}
